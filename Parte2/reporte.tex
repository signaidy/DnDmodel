
\documentclass[11pt]{article}
\usepackage[spanish]{babel}
\usepackage[utf8]{inputenc}
\usepackage[T1]{fontenc}
\usepackage{graphicx}
\usepackage{booktabs}
\usepackage{amsmath}
\usepackage{hyperref}
\usepackage{geometry}
\geometry{margin=1in}

\title{Modelado de Clasificación con UCI Poker Hand}
\author{(Tu nombre)}
\date{\today}

\begin{document}
\maketitle

\section{Introducción}
En este reporte analizamos el conjunto de datos \textit{Poker Hand} de UCI \cite{uci}, realizamos un análisis exploratorio de datos (EDA), construimos nuevas características explicativas y comparamos dos modelos de clasificación (Regresión Logística y Random Forest) con métricas adecuadas para un conjunto desbalanceado.

\section{Descripción del conjunto de datos}
Cada instancia representa una mano de póker con cinco cartas. Los atributos son diez columnas: cinco \textbf{trajes} ($S1..S5$, valores 1..4) y cinco \textbf{rangos} ($R1..R5$, valores 1..13). La etiqueta $y \in \{0,\ldots,9\}$ indica el tipo de mano.

\section{Análisis Exploratorio (EDA)}
\subsection{Distribución de etiquetas}
\begin{figure}[h]
  \centering
  \includegraphics[width=0.7\textwidth]{outputs/labels_distribution.png}
  \caption{Distribución de etiquetas (desbalance).}
\end{figure}

\subsection{Distribuciones marginales de rangos y trajes}
Incluimos histogramas por columna original como referencia (ver carpeta \texttt{outputs/}).

\section{Ingeniería de características}
Creamos variables que codifican patrones de póker: \emph{flush} (mismo traje), \emph{straight} (secuencia de rangos), multiplicidades de rangos (pares, trío, póker), número de valores únicos, estadísticos de resumen y brechas ordenadas entre rangos. Esto ayuda a capturar relaciones no lineales relevantes para la etiqueta.

\section{Modelos y métricas}
Entrenamos dos modelos:
\begin{itemize}
  \item \textbf{Regresión Logística (multinomial)} con \texttt{class\_weight=balanced} y estandarización.
  \item \textbf{Random Forest} con \texttt{class\_weight=balanced\_subsample}.
\end{itemize}

Usamos \textbf{StratifiedKFold (k=5)} y reportamos \textbf{Accuracy} y \textbf{Macro-F1}. Dado el desbalance, \textit{Macro-F1} es especialmente informativa.

\subsection{Resultados de validación cruzada}
Los promedios y desviaciones estándar se guardan en \texttt{outputs/cv\_summary.csv}. Puedes tabularlos aquí.

\subsection{Evaluación en holdout y análisis de errores}
\begin{figure}[h]
  \centering
  \includegraphics[width=0.49\textwidth]{outputs/confusion_logreg.png}
  \includegraphics[width=0.49\textwidth]{outputs/confusion_randomforest.png}
  \caption{Matrices de confusión en subconjunto de prueba (20\%).}
\end{figure}

\section{Importancia de características}
\begin{figure}[h]
  \centering
  \includegraphics[width=0.75\textwidth]{outputs/feature_importance_rf_top20.png}
  \caption{Top 20 importancias de Random Forest.}
\end{figure}

\section{Conclusiones}
Resumen comparativo de desempeño (trade-offs, tiempo de cómputo, interpretabilidad). Señalar posibles mejoras: más features (detección exacta de \emph{full house}, \emph{straight-flush}), modelos adicionales (Gradient Boosting, XGBoost), y manejo más fino del desbalance.

\bibliographystyle{plain}
\begin{thebibliography}{9}
\bibitem{uci}
Dua, D. and Graff, C. (2019). UCI Machine Learning Repository. \url{https://archive.ics.uci.edu/}.
\end{thebibliography}
\end{document}
